\maketitle

\begin{abstract}
\noindent Electrochemical cell measurements have been used to determine the thermodynamic properties of chemical reactions.\cite{1,2}
This experiment reinforces material often found in physical chemistry lectures. 
The experiments illustrates how thermodynamic properties such as the change in the standard Gibbs energy, \gibbs*{1}, and the equilibrium constant of a reaction are related to the emf, at different temperatures, of electrochemical cells in which the reaction takes place. 
The experiment is an application of a metal--insoluble salt electrode to determine the solubility of a sparingly-soluble salt, \ch{PbCl2}. 
\end{abstract}

\section{Introduction}
\label{sec:intro}


\section{Procedure}
\label{sec:procedure}
\todo{Rewrite this section to match other format.}

\subsection{Required Equipment}
\label{subs:required_equipment}

\begin{itemize}
	\item (2) \qty{100}{\mL} volumetric flasks
\end{itemize}

\subsection{Making the Stock Solutions}
\label{subs:stock_solutions}


\section{Data Analysis}
\label{sec:data_analysis}


\section{Lab Report Guidelines} % (fold)
\label{sec:lab_report_guidelines}

Your lab report should consist of the following parts:
\begin{description}
	\item[Title, Author and Date]
	\item[Introduction] Describe the experiment and expected results in a few sections. 
	\item[Experimental Theory] Reference this document, but include the requested derivation requested in the calculations section. In other words, your Theory section should be a reference (\eg, ``please see ``\pKa\ of Methyl Red'' in the CHEM355 Lab Manual for the theory) and a short derivation.
	\item[Experimental Procedure] This should be a very brief general outline of the procedure, written out as a paragraph or two. Give the make and model for any major instruments you used, as well as any important settings. The description should be thorough enough that another student can repeat your experiments. This means you must provide explicit volumes, weights, and temperatures. Use the past tense in all of your descriptions. Don't just copy the procedure from the manual, state what work \emph{you} performed. 
	\item[Results and Discussion] This should include an overview of the analyzed data and responses to the questions worked into a natural narrative. 
	Include your data in a \emph{tabular} format, including all the information necessary to repeat your calculations. 
	Include all graphs as figures (with captions). 
	Your graphs should include axes labels (with units). 
	Any fitted curves should be annotated with equations (slopes, intercepts) and uncertainties for each parameter. 
	Remember to use propagation of error rules in presenting the standard deviations in the final results. 
	Finally, also discuss the \emph{chemical significance} of the results. 
	The chemical significance can be addressed in several alternate ways:
	\begin{itemize}
		\item State why these results are useful and important, or
		\item State how this experiment and technique fit into the larger world of chemistry, or
		\item Discuss why someone might need to perform a study of this type.
	\end{itemize}
	\item[References] Include any external material you incorporated into this report. 
	\item[Appendix] At the very end of your report, include examples of any calculations that you did by hand. 
	Include any additional files and code that you used to generate your graphs.
\end{description}
